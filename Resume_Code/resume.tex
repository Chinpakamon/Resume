\documentclass{clean_cv}


% Add a BibTeX-style file encoding all of your publications to include here. You can export this from Zotero. Only include
% publications you want to appear here!
\addbibresource{publications.bib}

\author{Zakiev Vladislav}
\headlineposition{Software Engineer}

\begin{document}

\maketitle
% In this section, you can use any of the FontAwesome icons. The commands \faCenter and \faCenterStyle have been defined to properly center the icons
% when using the default font settings.
%
% You can use any of the icons listed in the fontawesome5 package documentation (https://ctan.math.utah.edu/ctan/tex-archive/fonts/fontawesome5/doc/fontawesome5.pdf)
% If you need to specify a specific style (as is done here for the address card), you should use the two-argument \faCenterCycle command
\begin{center}
\begin{tabular}{lll}
{\scriptsize}
    {\scriptsize}\faCenter{envelope} \color{blue}\href{mailto:zackieffv@yandex.ru}{zackieffv@yandex.ru}  & \faCenter{phone-alt} +7-917-894-07-09 
     \faCenter{github} \color{blue}\href{https://github.com/Chinpakamon}{Projects} &
     \faTelegram{\color{blue} \href{https://t.me/Chinpakamon}{Chinpakamon}}
\end{tabular}
\end{center}

\vspace{-1.5em}

\section{Education}

% The datetabular environment takes one argument, which is the width of the left date column. As seen here:
%   9em is a good choice for "dual-date" formats (e.g. Sep 2015 - Nov 2019).
%   4em is a good choice for month/year dates (Sep 2014).
%   2em is a good choice for year-only dates (as seen in the publications)
\begin{datetabular}{9em}
% This is just a tabular environment, for the most part. The dateentry command has been defined for
% convienence. It takes two arguments, the first is the date and the second is whatever you wish placed to the right.

\dateentry{Sep 2019 -- June 2023}{
\textbf{Saint Petersburg State University of Industrial Technologies and Design}

\textit{Bachelor of science in applied mathematics and computer science}
}

 

\end{datetabular}
\begin{datetabular}{9em}
\dateentry{Aug 2022 -- June 2023}{
\textbf{Yandex Practicum}

\textit{Learning Python Development and Frameworks}
}

\end{datetabular}

\begin{datetabular}{9em}
\dateentry{Sep 2023 -- Now}{
\textbf{ITMO University}

\textit{Programming and Internet technologies}
}
\end{datetabular}

\begin{datetabular}{9em}
\dateentry{Sep 2024 -- Now}{
\textbf{Backend Academy of T-Bank}

\textit{Study of programming, databases, application architecture.}
}

\end{datetabular}

\section{Experience}

\begin{datetabular}{9em}
\dateentry{Mar 2024 -- Jun 2024}{
\textbf{NIL IT}

\textit{Developed web applications, bots and supported existing company projects.}
}
\end{datetabular}

\begin{datetabular}{9em}
\dateentry{Aug 2024 -- Dec 2024}{
\textbf{LLC "Innovative Technologies"}

\textit{Developed asynchronous microservices using FastAPI and asyncio.}
}
\end{datetabular}
 
\section{Skills}

\textbf{Technical}
 \begin{itemize}
   \item Python, Golang, Django, DRF, FastAPI, asyncio, PostgreSQL, SQLite, SQL, Yandex.Cloud, Docker, Nginx, Linux, Git
   \end{itemize}
   \vspace{0.5cm}
   
\textbf{Languages}
 \begin{itemize}
   \item English(B1), Russian(native)
   \end{itemize}
   \vspace{0.5cm}
\section{Projects}

\textbf{Usermanagment microservice}
\begin{itemize}
    \item  An asynchronous microservice designed to interact with the user, provide access depending on the role, change data, etc.
    \item used Python, FastAPI, PostgreSQL, Redis, RabbitMQ, Poetry, Docker and Docker Compose, Pydantic Settings, Linters: black, isort, ruff.
    \end{itemize}
    \vspace{0.5cm}

\textbf{Novis}
\begin{itemize}
    \item  Web application for accounting and interaction with foreign students at the university, working with them, delivering news and documents within the university
    \item used Python, Django, Django Rest Framework, PostgreSQL, Docker and Docker Compose, etc. 
    \end{itemize}
    \vspace{0.5cm}

\textbf{Foodgram}\href{https://github.com/Chinpakamon/foodgram-project-react}{\color{blue}\faCenter{github}}
 \begin{itemize}
   \item  On this service, users can post recipes, subscribe to other posts, add favorite recipes to their Favorites list, and download a summary of the products they need before going to the store to prepare one or more selected dishes
   \item used Python, Django, DRF, Gunicorn, Nginx, Docker, Docker-compose, PostgreSQL, drf-simlejwt
   \end{itemize}

\end{document}

