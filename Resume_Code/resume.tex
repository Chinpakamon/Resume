\documentclass{clean_cv}


\addbibresource{publications.bib}

\author{Zakiev Vladislav}
\headlineposition{Software Engineer}

\begin{document}

\maketitle

\begin{center}
\begin{tabular}{lll}
{\scriptsize}
    {\scriptsize}\faCenter{envelope} \color{blue}\href{mailto:zackieffv@yandex.ru}{zackieffv@yandex.ru}  & \faCenter{phone-alt} +7-917-894-07-09 
     \faCenter{github} \color{blue}\href{https://github.com/Chinpakamon}{Projects} &
     \faTelegram{Chinpakamon}
\end{tabular}
\end{center}

\vspace{-1.5em}

\section{Education}

\begin{datetabular}{9em}

\dateentry{Sep 2019 -- June 2023}{
\textbf{Saint Petersburg State University of Industrial Technologies and Design}

\textit{Bachelor of science in applied mathematics and computer science}
}

 

\end{datetabular}
\begin{datetabular}{9em}
\dateentry{Aug 2022 -- June 2023}{
\textbf{Yandex Practicum}

\textit{Python Developer}
}

\end{datetabular}

\begin{datetabular}{9em}
\dateentry{Sep 2023 -- Now}{
\textbf{ITMO University}

\textit{Programming and Internet technologies}
}
\end{datetabular}



 
\section{Skills}

\textbf{Technical}
 \begin{itemize}
   \item Python, Django, PostgreSQL, SQLite, SQL, Yandex.Cloud, Docker, Nginx, Linux, Git
   \end{itemize}
   \vspace{0.5cm}
   
\textbf{Languages}
 \begin{itemize}
   \item English(B1), Russian(native)
   \end{itemize}
   \vspace{0.5cm}
\section{Projects}

\textbf{Foodgram}\href{https://github.com/Chinpakamon/foodgram-project-react}{\color{blue}\faCenter{github}}
 \begin{itemize}
   \item  On this service, users can post recipes, subscribe to other posts, add favorite recipes to their Favorites list, and download a summary of the products they need before going to the store to prepare one or more selected dishes
   \item used Python, Django, DRF, Gunicorn, Nginx, Docker, Docker-compose, PostgreSQL, drf-simlejwt
   \end{itemize}
   \vspace{0.5cm}
   
   \textbf{Telegram Bot}\href{https://github.com/Chinpakamon/homework_bot}{\color{blue}\faCenter{github}}
 \begin{itemize}
   \item  The bot receives data via API from the Yandex-Practicum website. Checks the result and compares it with the previous query. Generates a response and sends it to the Telegram chat.
   \item used Python, python-telegram-bot 13.7
   \end{itemize}
   \vspace{0.5cm}
   
   \textbf{Api YaMDB(Group project)}\href{https://github.com/Chinpakamon/api_yamdb/tree/master}{\color{blue}\faCenter{github}}
 \begin{itemize}
   \item  The project provides access to the YaMDb service via an API. My task was to describe models, view and endpoints for products, categories, genres, as well as implement data import from csv files, etc.
   \item used Python, Django, Django Rest Framework, Simple JWT, SQLite3
   \end{itemize}

\end{document}
